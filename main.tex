%%
%% Beginning of file 'sample62.tex'
%%
%% Modified 2018 January
%%
%% This is a sample manuscript marked up using the
%% AASTeX v6.2 LaTeX 2e macros.
%%
%% AASTeX is now based on Alexey Vikhlinin's emulateapj.cls 
%% (Copyright 2000-2015).  See the classfile for details.

%% AASTeX requires revtex4-1.cls (http://publish.aps.org/revtex4/) and
%% other external packages (latexsym, graphicx, amssymb, longtable, and epsf).
%% All of these external packages should already be present in the modern TeX 
%% distributions.  If not they can also be obtained at www.ctan.org.

%% The first piece of markup in an AASTeX v6.x document is the \documentclass
%% command. LaTeX will ignore any data that comes before this command. The 
%% documentclass can take an optional argument to modify the output style.
%% The command below calls the preprint style  which will produce a tightly 
%% typeset, one-column, single-spaced document.  It is the default and thus
%% does not need to be explicitly stated.
%%
%%
%% using aastex version 6.2
\documentclass[modern]{aastex62}



\begin{document}
%\maketitle
\title{Dissertation Summary}
\author{Krzysztof Suberlak}
\affil{University of Washington, Seattle}

\date{ \today
}

\begin{abstract}
The dissertation summary.... 
\end{abstract}

\section*{Prompt}
"Provide the 10 page dissertation summary to your committee (with a cc: to Christine for
departmental records). The summary includes the scientific context for your proposed
dissertation, a description of the proposed research and a likely timeline for completion of
major milestones."

\section{Scientific Context}

Quasars are some of the brightest sources of radiation in the Universe.  Termed initially as Quasi-Stellar Objects, they are sources of light that looks like a star - an unresolved point source of light to a naked eye or a small telescope. However a much larger telescope size, or a spectrograph to split the incoming light into the individual components of the spectrum, reveal a pattern unlike any star. Indeed, the light from this objects is non-thermal in nature - definitely not originating from a hot material radiating away heat from the surface of a spheroid - like any other star in the Universe. Yet, when seen with a much larger telescope, allowing more time to gather the light, we would realize that there is also a faint glow of a host galaxy around the intense star-like source of light.  Yet the galaxy's faint glow is so much dimmer and more diffuse than the quasar light, that since the serendipituous discovery of quasars by Maarten Schmidt in 1963 \citep{schmidt1963, richards2009a}, we have not seen their host galaxies until decades later \citep{hooper1997, boyce1999, lehnert1999}.


Early on it was realized that the quasar light that overwhelms the host galaxy in its intensity originates from the hot accretion disk \citep{oke1965, burbidge1967}.  Such disk of a hot material feeds a supermassive black hole (SMBH), lying in the center of a host galaxy. We have seen light from similar accretion disks before from more local galaxies that host an actively accreting SMBH  - an active galactic nucleus, or an AGN (see overview in \cite{netzer2013}). This phase of galaxy's life is characterized by an accretion of gas onto the SMBH. The gas supply could come from a merger of two (or more) galaxies. Indeed, we have seen dual AGNs, offset AGNs, etc. \citep{kormendy2013, muller2016}. With AGNs we often see the radio lobes of matter that while infalling onto an SMBH was funnelled outwards with very strong magnetic field via Blandford-Znajek mechanism \citep{blandford1977}. We also see various features in its spectrum that come from hot clouds of gas orbiting the SMBH near (so-called Broad-Line Region), or far (the Narrow-Line Region). Since an accretion disk is not spherically-symmetric (and neither is the doughtnut-shaped cloud of gas and dust surrounding it), the AGN would look slightly different depending on an observing angle  \citep{marin2017, lawrence2016a} (as well as gas content, etc \cite{veilleux2016}) . The strong concentration of gas may also trigger the intence star formation (LINERs) \citep{maiolino2003, mingo2016,  hernandez2016}. The dense  dust may obscure part of the disk emission (LIRGs, Sey2) \citep{symeonidis2017, lamassa2017}. AGNs are as diverse as are the possibilities of evolution of a galaxy, as it accretes gas, and merges with the nearby galaxies in the process of structure formation. Since it is not surprising that when the Universe was denser at earlier times (higher redshifts), the mergers of galaxies were more frequent, and the supply of gas higher, we would expect that the AGNs of the past (or earlier times, higher redshifts), are also more powerful than those at work 'today', or in the nearby universe. This is indeed the case - we find that quasars are just much more distant, more luminous AGNs. It is only the sheer distance (and intrinsic brightness) that resulted in different features observed at first (eg. point-like accretion disk emission for a quasar, rather than the extended radio emission from AGN's radio-lobes). 


%Such active galactic nuclei - AGN - have been observed in a more local universe, where we can distinguish the intense radiation of  jets of matter directed away from the disk, radio-lobes of heated gas emitted away from the central maelstrom, and the host galaxy. Quasars are orders of magnitude brighter than local AGN, and for this reason (and their cosmological distance) have traditionally deserved this separate nomenclature.

Quasars are an excellent probe of the evolution of the universe over time. Present across vast range of distances (redshifts), quasars are witnesses to the conditions in the universe from the earliest times of first galaxies, to later mergers of galaxies in the center  of a cluster.  The redshift distribution of quasars peaks at the redshift $\approx$2, and they have been seen all the way from redshift 0.1 to 9 \citep{paris2017}. As quasar light passes through other galaxies and clouds of gas before it reaches us, they leave their imprint  in the form of absorption lines and troughs.  For this reason quasar spectra are used  for studies of the intergalactic medium (\cite{prochaska2014} and references therein),  He II reionization history \citep{khrykin2017} , Damped Lyman $\alpha$ Absorbers (\cite{wolfe2005} for overview, also see more recent \cite{parks2018} about the effort to find DLA by deep learning of QSO spectra, and \cite{murphy2016} for study of  dust content in DLA from SDSS).  Quasars quite literally shine light on the epoch of reionization - the moment when the universe, full of neutral Hydrogen and Helium from the Big Bang, became ionized \citep{glikman2011, masters2012, ross2013}. As  \cite{alsayyad2016} points out, the shape of the Quasar Luminosity Function (QLF) is connected to the SMBH growth, as well as the cosmic reionization. As quasars accompany a phase of a rapid growth of an SMBH, studying their duty cycle helps properly understand the growth and co-evolution of  galaxies SMBH \citep{schawinski2012}.  QLF is also directly related to the evolution of the galactic Initial Mass Function \citep{mcgreer2013}. 

Quasars are very interesting objects in their own right. An accretion disk is an extreme environment: it is a disk of viscously heated matter inspiralling onto an SMBH \citep{ruan2017}. The disk by itself is merely a few times bigger than the solar system (\citep{mudd2017} shows with reverberation mapping studies that for the majority of their sample disk sizes at 2500 angstroms are between 1-10 light days - compare that to the orbit of pluto, of approximately 0.2277 light days.) The disk itself, due to a relatively small size compared to the galaxy cannot be resolved with even the largest currently operating telescope (eg. TMT, GMT - cite biggest telescopes so far).  Yet, emission from the accretion disk overwhelms all the light from the stars of the host galaxy.   In case of quasars, the accretion disk is so much brighter than all the stars of the host galaxy, that initially one only sees the light that appears to all come from one point in space - exactly like in case of a star. Hence the light from an accretion disk appears point-like : quasi-stellar. However,  if one allows to collect the light for much longer, with bigger collecting area  we begin to see the stellar light from the galaxy hosting the active nucleus \citep{hutchings2002,kotilainen2013, falomo2014, liuzzo2016, bayliss2017}. 

Given the inherent dynamics of the AGN or quasar phase of galaxy's lifetime, it makes sense that as the gas supply and the SMBH environment varies, the brightness should change over time both for an accretion disk, as well as any absorption/emission features caused by the orbiting clouds.\cite{stern2017, schawinski2015}.  Since for a quasar the emission from the accretion disk is dominant compared to the stellar light from the host galaxy, we can assume that all the coherent variability  in the quasar light curve is related to the accretion disk variability . The quasar brightness varies over timescales from millions of years to hours, variability on each timescale tied to a different physical mechanism.  The longest timescales of millions of years (postulated but not directly measured) correspond to the changes in overall accretion pattern as the galactic merger that started the quasar phase progresses.  Indeed, it is now thought that quasars are just an episode ( or multiple episodes) in life of almost any galaxy \citep{alexander2012, kormendy2013}. Progressing to shorter (more directly measurable) timescales of years, the brightness of a  quasar may change due to a change in the strength of an underlying continuum emission - in other words, changes in gas supply (see eg. Changing-Look AGNs, and discovered recently Changing-Look Quasars- \cite{ruan2016, macleod2016, graham2017}). There is also variability on medium timescales of weeks to months that can be easily observed in a quasar light curve. Indeed - it was variability on these medium timescales  in the optical light that first hinted at the relatively small size of the region emitting the light (if the brightness of an entire disk changes, two parts of the disk must be separated by less than the light travel time.) \cite{mudd2017, blackburne2011, morgan2010}. Quasar light curve in the optical looks like a random noise - there seems to be no overall pattern underlying the changes in brightness. Under closer inspection it turns out that the signal is stochastic - close to chaotic, but with an inherent order. Stochastic variability can be characteristic of thermal processes, such as spread of instabilities in the disk - 'hot spots', driven by the magnetohydrodynamic (MHD) processes \citep{kelly2007, zu2013, kozlowski2016a}. Therefore, the parameters describing quasar variability on these medium-long timescales relate to the physical parameters of the accretion : disk extent, its viscosity and temperature, strength of magnetic field. On shortest timescales of hours, only recently probed by Kepler, the light curves are also consistent with  the MHD instabilities\citep{dexter2011, kasliwal2015a, aranzana2018, smith2018}.


Recent reverberation mapping studies largely confirm this picture \citep{sun2015}. One way of describing the variation in brightness over time is to plot the signal against time - that is the classic light curve.  A light curve can be further characterized by a structure function (SF), that describes the spread in magnitude differences between pairs of points as a function of time separation. SF tells us how much correlation there is between different parts of light curve, and it is also related to the power spectral density. SF can be seen as a mirror image of the Power Spetral Density (PSD), in a sense that SF is a function of time differences, while PSD is a function of frequency - the inverse of  time differences. Processes that have a flat PSD - having an equal power across the entire frequency spectrum -  are described as "white" noise, like white light that includes all colors. To further this analogy, stochastic processes that can be described by a simple power law, and yet depart from "white" noise, are termed  "colored" noise, depending on the relative amplitude of PSD at various frequencies. Thus we have "red", "pink", or "blue" noise   (see Appendix  C in \cite{kasliwal2017} for further details). ( Also, great illustration of how a light curve looks like depending on an exponent of a PSD is found in \cite{macleod2010}, Fig. 9 ). 


Over the last decade we have significantly improved our understanding of quasars, and new variability-based clasification techniques \citep{fan2001, richards2006, kozlowski2010, palanque2011, macleod2011, graham2014, alsayyad2016, ruan2017} have yielded an unprecedented number of quasars that allows a statistical study. New quasar catalogues (eBOSS, etc) allowed  to further constrain the Quasar Luminosity Function \citep{ross2013, myers2015, palanque2016}. Classically quasars were selected using color cuts,  which mainly targets quasars at redshifts around 0.5 < z < 2.5 (Large Bright Quasar Survey, \cite{hewett1995}, the 2dF QSO Redshift Survey \cite{croom2004}, and the 2dF-SDSS LRGand QSO Survey \cite{croom2009}).  In addition to color space, SDSS I,II selected quasars via radio matches to the FIRST survey, or x-ray matches to ROSAT survey  - see introduction of \citep{myers2015}. Since then other classification parameters have been tested, including radio source matching \citep{mcgreer2009}, near-IR color cuts \citep{banerji2012},  radio + near-IR \citep{glikman2012}, mid-IR \citep{stern2005, richards2009a, stern2012}, X-rays \citep{trichas2012},  midIR + X-rays \citep{lacy2004, hickox2007, hickox2009}, variability \citep{schmidt2010, butler2011, macleod2011, palanque2011}. There are some studies that combined color and variability information, eg. \cite{tie2017, peters2015, sesar2007}. 

 
A better understanding of how well the Damped Random Walk model can be used to model quasar variability is crucial for using DRW model best-fit parameters to characterize and classify quasars. It is a relatively recent field : \cite{kelly2009} showed that DRW parameters can be linked to the physics of the accretion disk. Since \cite{kozlowski2010} proposed the DRW model for quasar selection, \cite{macleod2010} moved the field of quasar variability studies from a handful of objects to a proper statistical study, fitting over 9 000 quasars in SDSS Stripe 82 with the DRW model \cite{graham2017,kozlowski2017a} showed that when using the  DRW model, the length of light curve is the most important predictor of biases for the best-fit parameters. 

His main conclusion is that when the light curve length (baseline) is insufficient( less than 10 times the intrinsic timescale ), then the bias will be larger than a few \%. He showed that for a light curve of length similar to the intrinsic timescale, we expect the bias in retrieved timescale of a factor of 3, which we confirm in our independent study.  We find with Celerite that a longer basline can decrease the bias, which implies that the extended light curve can help improve the results of \cite{macleod2011}.  

Despite these advances, there are several crucial questions related to quasar variability that remain open. First, despite general agreement that the DRW is a good first-order description of quasar variability (\cite{zu2011,kozlowski2010}), there have been some inquiries to the fidelity of these descriptions on all timescales \cite{zu2013, kasliwal2017, sartori2018}. In particular, \cite{graham2014} reported  a detection of characteristic restframe timescale of 54 days, which has not been detected before.  We  aim to reconcile the CRTS data with the SDSS and PTF data for the same S82 quasars. It was surprising given that other studies more well tuned to short timescales did not report such findings ( especially studies involving Kepler K2  AGN light curves with extremely good cadence - see \cite{kasliwal2015a, aranzana2018, smith2018}).

Second, although \cite{kelly2011} showed that the DRW is a good description of the quasar variability, \cite{kozlowski2017a} suggested that with the method he employed there are biases in retrieved parameters. We want to evaluate the existing tools for fitting light curves (eg. the implementation of \cite{rybicki1995} method, JAVELIN of \cite{zu2011}, George, Celerite, etc. ), and find the best tool for future studies given their scalability (speed), memory usage, bias, to be prepared for fitting of billions of light curves in the LSST era. 

Finally, we consider whether we can improve the "classical" method of light curve classification by using combination kernel Gaussian Processes. In the standard paradigm,  to compare models A and B (eg. sinusoidal oscillation vs DRW), one needs to fit  - optimize the cost function for model A and model B,  calculate metric describing the 'goodness of fit' (eg. $\chi^{2}_{DOF}$), and then choose a better fitting model based on that metric. We investigate a Bayesian method of fitting to the data a combination kernel Gaussian process, where one obtains simultaneously maximum a-posteriori estimates for  amplitudes of component kernels. 

\section{Proposed Research}

\subsection{Discrepant quasar timescales}
To answer the first question, we presented in \cite{suberlak2017} an improved photometric error analysis for the 7 100 CRTS (Catalina Real-Time Transient Survey) optical light curves for quasars from the SDSS (Sloan Digital Sky Survey) Stripe 82 catalogue. The SDSS imaging survey has provided a time-resolved photometric data set, which greatly improved our understanding of the quasar optical continuum variability. Data for monthly and longer time-scales are consistent with a damped random walk (DRW). Newer data obtained by CRTS provided puzzling evidence for enhanced variability, compared to SDSS results, on monthly time-scales. Quantitatively, SDSS results predicted about 0.06 mag root-mean-square (rms) variability for monthly time-scales, while CRTS data showed about a factor of 2 larger rms, for spectroscopically confirmed SDSS quasars. Our analysis has successfully resolved this discrepancy as due to slightly underestimated photometric uncertainties from the CRTS image processing pipelines. As a result, we found that the correction for observational noise was too small and the implied quasar variability was too large. The CRTS photometric error correction factors, derived from detailed analysis of non-variable SDSS standard stars that were re-observed by CRTS, are about 20–30 \%, and result in reconciling quasar variability behaviour implied by the CRTS data with earlier SDSS results. An additional analysis based on independent light curve data for the same objects obtained by the Palomar Transient Factory provides further supported for this conclusion. In summary, the quasar variability constraints on weekly and monthly time-scales from SDSS, CRTS and PTF surveys are mutually compatible, as well as consistent with DRW model, as described in \citep{suberlak2017}. 



\subsection{Extending the light curve baseline, using the right tools}
We first confirm the scaling relations by \cite{kozlowski2017a} by testing the retrieval of simulated light curve parameters with Celerite .  Using a similar MAP-based method we find with Celerite that the same bias in output light curve is found, and is primarily dependent on light curve length in relation to input time scale. Using a fixed light curve length of $t_{exp} = 8$ years we simulate 100 different input time scales $\tau_{in}$ ,  where  $\rho = \tau_{in} / t_{exp} \in   { 0.01 : 15}$.  The bias is not affected by different cadence (we tested SDSS and CRTS cadences of N=80 or N=445 points).


Having established that extending the light curve baseline decreases the bias in retrieved DRW time scales we set out to improve on relations measured by \cite{macleod2011, hernitschek2016} adding more data points to SDSS and PanSTARRS light curves \citep{chambers2011}. We do that by augmenting the existing light curves with additional data from PanSTARRS DR2 \citep{flewelling2018} , CRTS \citep{drake2009}, and PTF \citep{rau2009}.  We start from the SDSS  DR7 near-simultaneous ugriz photometry for S82 quasars\citep{schneider2008}.  To use combined datasets we cross-match the catalogs, and find a common photometric solution using the S82 standard stars (PTF and CRTS use white light, \cite{djorgovski2011a}). Querying CRTS DR2 database B.Sesar obtained CRTS white light lightcurves for the S82 DR7 quasars.  We obtained the PTF light curves from the PTF IRSA database.  C. MacLeod provided  the PS1  grizy  light curves matched to positions from DR7 catalog.  We validate data quality by comparing mean CRTS magnitude vs mean SDSS magnitude, and mean PS1 g-band to mean PS1 r-band. 

We first simulate 10 000 DRW light curves at the SDSS cadence, and we test how adding more data (corresponding to PS1, PTF, CRTS datapoints) affects the fidelity of the retrieved to input parameters. Confirming that adding more data at these cadences improves the fit, we fit the real data with the DRW model, and using the spectroscopic information about the quasars (such as black hole mass, luminosity - \cite{kelly2013}) we reconsider relations tested by \cite{macleod2011}. 

We also test how selecting photometry from only a subset of surveys $\tau_{PS1}$, $\tau_{(PS1+SDSS)}$,  $\tau_{SDSS}$.  We also revisit all relations using only bright quasars with   $\tau_{\langle mag\rangle<19}$.

\subsection{Improving the classical model fitting in the  framework of Gaussian processes}

Classical model selection involves specifying a loss function that can be used to find best-fitting parameters for a given model. A suitable metric, such as $\chi^{2}$, is used to  describe the scatter of data around the model. Choosing a better model involves comparing the values of the loss function. A different approach is to simultaneously find the maximum a-posteriori solution for a model that combines the two. Such possibility is afforded by Gaussian Process with combination kernels,  implemented in Celerite  \citep{foreman2017}. For the last decade Gaussian Processes have become more well known, employing a class of functions that  are characterized by covariance between pairs of points in the dataset \citep{foreman2017}.  The combination kernel technique has been successfully employed by  \cite{angus2018}  to model  quasi-periodic oscillation of star spots on top of the sinusoidally varying signal due to stellar rotation. Success of this approach furthers our interest in employing this methodology to mine the data for a sinusoidal signal superimposed on top of the DRW, such as is the case for the binary SMBH \citep{charisi2018}. 

We first test the combination kernel method, simulating a DRW light curve (parametrized by asymptotic amplitude $SF_{\infty}$, and characteristic timescale $\tau$ ) , with  added sinusoidal modulation (parametrized by amplitude A, period P). With input characteristic timescale of $\tau  = 100 $ days, and regular sampling every dt = 5 days  , we explore regimes from $A << SF_{\infty}$,  to $A ~ SF_{\infty}$ , and from $P << \tau$, to $P >> \tau$ :  $A \in { 0.01,  0.1,  0.25,  0.5, 0.75 } SF_{\infty}  \times  P \in { 0.25,  1 ,  4 } \tau  $. We aim to find whether we can distinguish between a pure stochastic signal, and stochastic signal with sinusoidal modulation. We then test the combination kernel Gaussian process on spectroscopically confirmed SDSS S82 quasars, and finally find  the combination kernel amplitudes for all SDSS S82 light curves.


\section{Timeline}


%%%%%%%%%%%%%%%%%%%% REFERENCES %%%%%%%%%%%%%%%%%%
\bibliographystyle{mnras}
\bibliography{referencesMaster}

\end{document}