%%
%% Beginning of file 'sample62.tex'
%%
%% Modified 2018 January
%%
%% This is a sample manuscript marked up using the
%% AASTeX v6.2 LaTeX 2e macros.
%%
%% AASTeX is now based on Alexey Vikhlinin's emulateapj.cls 
%% (Copyright 2000-2015).  See the classfile for details.

%% AASTeX requires revtex4-1.cls (http://publish.aps.org/revtex4/) and
%% other external packages (latexsym, graphicx, amssymb, longtable, and epsf).
%% All of these external packages should already be present in the modern TeX 
%% distributions.  If not they can also be obtained at www.ctan.org.

%% The first piece of markup in an AASTeX v6.x document is the \documentclass
%% command. LaTeX will ignore any data that comes before this command. The 
%% documentclass can take an optional argument to modify the output style.
%% The command below calls the preprint style  which will produce a tightly 
%% typeset, one-column, single-spaced document.  It is the default and thus
%% does not need to be explicitly stated.
%%
%%
%% using aastex version 6.2
\documentclass[modern]{aastex62}



\begin{document}
%\maketitle
\title{Dissertation Summary}
\author{Krzysztof Suberlak}
\affil{University of Washington, Seattle}

\date{ \today
}

\begin{abstract}
The dissertation summary.... 
\end{abstract}

\section*{Prompt}
"Provide the 10 page dissertation summary to your committee (with a cc: to Christine for
departmental records). The summary includes the scientific context for your proposed
dissertation, a description of the proposed research and a likely timeline for completion of
major milestones."

\section{Scientific Context}

Quasars are some of the brightest sources of radiation in the Universe. 

What is a quasar? Quasi-Stellar Object - a source of light that appears like an unresolved point source, a star. Yet it does not look like any star when investigated under a spectroscope - the spectrum is very non-thermal,  unlike any blackbody. When observed with a larger telescope, and longer exposure time, we may gather enough light to realize that there is also a faint glow of a host galaxy  surrounding it - so much fainter than the quasar light that it was not realized until few decades after a serendipitous discovery of the first quasar. A quasar is an emission from the hot accretion disk surrounding a supermassive black hole in the center of the host galaxy.  Such active galactic nuclei - AGN - have been observed in a more local universe, where we can distinguish the intense radiation of  jets of matter directed away from the disk, radio-lobes of heated gas emitted away from the central maelstrom, and the host galaxy. Quasars are orders of magnitude brighter than local AGN, and for this reason (and their cosmological distance) have traditionally deserved this separate nomenclature.

Because quasars are present across the observable universe, from the earliest newly-forming galaxies in the recently reionized universe, to those galaxies that have already formed billions of years earlier, they are an excellent probe of how the universe and its content evolved over time. 


Quasars are used as very bright light houses that shine through other galaxies, or clouds of gas, and both leave imprint in the light of the quasar  (studies of IGM : \cite{prochaska2014} and references therein,  He II reionization history \cite{khrykin2017} , Damped Lyman $\alpha$ Absorbers - see \cite{wolfe2005} for overview, also more recent \cite{parks2018} about the effort to find DLA by deep learning of QSO spectra, and \cite{murphy2016} for study of  dust content in DLA from SDSS) .  

Quasars shine light on the epoch of reionization - the moment when the universe, full of neutral Hydrogen and Helium from the Big Bang, became ionized. As  \cite{alsayyad2016} points out, the shape of the Quasar Luminosity Function is connected to the SMBH growth, as well as the cosmic reionization. QLF contains information about the amount of quasar contribution in reionizing the universe \citep{glikman2011, masters2012, ross2013}






Since quasars accompany a phase of a rapid growth of a supermassive black hole, we need to study their duty cycle to properly understand the growth of galaxies, and AGN feedback. Quasar luminosity function is directly related to the evolution of the galactic initial mass function \citep{mcgreer2013}. 

An accretion disk is an extreme environment. It is a disk of viscously heated matter inspiralling onto a supermassive black hole \citep{ruan2017}. The disk itself is merely a few times bigger than  the solar system -  \citep{mudd2017} uses reverberation mapping to show that for the majority of their sample disk sizes at 2500 angstroms are between 1-10 light days. Compare that to the orbit of pluto, of approximately 0.2277 light days. The disk itself due to such small size (compared to the galaxy itself) cannot be resolved with a telescope.  Yet, the emission of the accretion disk overwhelms all the light from the stars of the host galaxy.   In case of quasars, the accretion disk is so much brighter than all the stars of the host galaxy, that initially one only sees the light that appears to all come from one point in space - exactly like in case of a star. Hence the light from an accretion disk appears point-like : quasi-stellar. However,  if one allows to collect the light for much longer, with bigger collecting area  we begin to see the stellar light from the galaxy hosting the active nucleus \citep{hutchings2002,kotilainen2013, falomo2014, liuzzo2016, bayliss2017}. 

Although we cannot resolve the accretion disk, it turns out that thanks to its dynamic behavior  - the matter actively accreting onto the central supermassive black hole - its brightness varies over time, on a diverse range of timescales \citep{schawinski2015}. This accretion disk flicker can be linked to the physical properties of the disk : its extent, viscosity, and processes driving its evolution, which helps understand the process of growth of supermassive black holes.  Since for a quasar the emission from the accretion disk is dominant compared to the stellar light from the host galaxy, we can assume that all the coherent variability  in the quasar light curve is related to the accretion disk variability . Indeed, the discovery that the quasar light can vary significantly over a period of weeks led to implication that the size of the accretion disk cannot be larger than few light weeks, due to causality. 

Now it is generally understood that the characteristic timescale of this stochastic  variability can be related to the thermal timescales of the disk \citep{kelly2007, zu2013, kozlowski2016a}. Recent reverberation mapping studies largely confirm this picture \citep{sun2015}




We find that the stochastic variability of quasar light over time can be mathematically described as a Damped Random Walk. 



One way of describing the variation in brightness over time is to plot the signal against time - that is the light curve.  A light curve can be further characterized by a structure function, that describes the spread in magnitude differences between pairs of points as a function of time separation. Structure Function tells us how much correlation there is between parts of light curve separated by a given time.  

Structure function is also related to the power spectral density. SF can be seen as a mirror image of PSD, in a sense that SF is a function of time differences, while PSD is a function of frequency , which is the inverse of  time differences. 

Processes that have a flat PSD - having an equal power across the entire frequency spectrum -  are described as "white" noise, like white light that includes all colors. To further this analogy, stochastic processes that can be described by a simple power law, and yet depart from "white" noise, are termed  "colored" noise, depending on the relative amplitude of PSD at various frequencies. Thus we have "red", "pink", or "blue" noise   (see Appendix  C in \cite{kasliwal2017} for further details). ( Also, great illustration of how a light curve looks like depending on an exponent of a PSD is found in \cite{macleod2010}, Fig. 9 ). 



Over the last decade we have significantly improved our understanding of quasars, and new variability-based clasification techniques \citep{fan2001, richards2006, kozlowski2010, palanque2011, macleod2011, graham2014, alsayyad2016, ruan2017} have yielded an unprecedented number of quasars that allows a statistical study. The new quasar catalogues (eBOSS, etc) allowed  to further constrain the Quasar Luminosity Function \citep{ross2013, myers2015, palanque2016}.

Classically quasars were selected using color cuts,  which mainly targets quasars at redshifts around 0.5 < z < 2.5 (Large Bright Quasar Survey, \cite{hewett1995}, the 2dF QSO Redshift Survey \cite{croom2004}, and the 2dF-SDSS LRGand QSO Survey \cite{croom2009}). 


In addition to color space, SDSS I,II selected quasars via radio matches to the FIRST survey, or x-ray matches to ROSAT survey  - see introduction of \citep{myers2015}. Since then other classification parameters have been tested, including radio source matching \citep{mcgreer2009}, near-IR color cuts \citep{banerji2012},  radio + near-IR \citep{glikman2012}, mid-IR \citep{stern2005, richards2009a, stern2012}, X-rays \citep{trichas2012},  midIR + X-rays \citep{lacy2004, hickox2007, hickox2009}, variability \citep{schmidt2010, butler2011, macleod2011, palanque2011}. 

 
A better understanding of how well the Damped Random Walk model can be used to model quasar variability is crucial for using DRW model best-fit parameters to characterize and classify quasars. It is a relatively recent field : Kelly+2009 showed that DRW parameters can be linked to the physics of the accretion disk.  Then  MacLeod+2011 moved the field of quasar variability studies from a handful of objects to a proper statistical study, fitting over 10 000 quasars in SDSS Stripe 82 with DRW model. This was followed by others, in particular Kozlowski+2017, who showed that in DRW model the length of light curve is the most important predictor of biases in best-fit parameters.  His main conclusion is that when the light curve length (baseline) is not long enough ( less than 10 times the intrinsic timescale ), then the bias will be larger than a few \%. He showed that for a light curve of length similar to the intrinsic timescale, we expect the bias in retrieved timescale of a factor of 3, which we confirm in our independent study.  We find with Celerite that a longer basline can decrease the bias, which implies that the extended light curve can help improve the results of MacLeod+2011.  

Despite these advances, there are several crucial questions related to quasar variability  that we aim to answer.   

First, despite general agreement that the DRW is a good first-order description of quasar variability (\cite{zu2011,kozlowski2010, }), there have been some inquiries to the fidelity of these descriptions on all timescales \cite{zu2013, kasliwal2017a, kasliwal2017}. In particular, \cite{graham2014} reported  a detection of characteristic restframe timescale of 54 days, which has not been detected before.  We  aim to reconcile the CRTS data with the SDSS and PTF data for the same S82 quasars.

Second, although \cite{kelly2011} showed that the DRW is a good description of the quasar variability, \cite{kozlowski2017a} suggested that with the method he employed there are biases in retrieved parameters. We want to evaluate the existing tools for fitting light curves (eg. the implementation of \cite{rybicki1995} method, JAVELIN of \cite{zu2011}, George, Celerite, etc. ), and find the best tool for future studies given their scalability (speed), memory usage, bias, to be prepared for fitting of billions of light curves in the LSST era. 

Finally, we consider whether we can improve the "classical" method of light curve classification by using combination kernel Gaussian Processes. In the standard paradigm,  to compare models A and B (eg. sinusoidal oscillation vs DRW), one needs to fit  - optimize the cost function for model A and model B,  calculate metric describing the 'goodness of fit' (eg. $\chi^{2}_{DOF}$), and then choose a better fitting model based on that metric. We investigate a Bayesian method of fitting to the data a combination kernel Gaussian process, where one obtains simultaneously maximum a-posteriori estimates for  amplitudes of component kernels. 


\section{Proposed Research}

To answer the first question, we presented in \cite{suberlak2017} an improved photometric error analysis for the 7 100 CRTS (Catalina Real-Time Transient Survey) optical light curves for quasars from the SDSS (Sloan Digital Sky Survey) Stripe 82 catalogue. The SDSS imaging survey has provided a time-resolved photometric data set, which greatly improved our understanding of the quasar optical continuum variability: Data for monthly and longer time-scales are consistent with a damped random walk (DRW). Recently, newer data obtained by CRTS provided puzzling evidence for enhanced variability, compared to SDSS results, on monthly time-scales. Quantitatively, SDSS results predict about 0.06 mag root-mean-square (rms) variability for monthly time-scales, while CRTS data show about a factor of 2 larger rms, for spectroscopically confirmed SDSS quasars. Our analysis has successfully resolved this discrepancy as due to slightly underestimated photometric uncer- tainties from the CRTS image processing pipelines. As a result, the correction for observational noise is too small and the implied quasar variability is too large. The CRTS photometric error correction factors, derived from detailed analysis of non-variable SDSS standard stars that were re-observed by CRTS, are about 20–30 per cent, and result in reconciling quasar vari- ability behaviour implied by the CRTS data with earlier SDSS results. An additional analysis based on independent light curve data for the same objects obtained by the Palomar Transient Factory provides further support for this conclusion. In summary, the quasar variability con- straints on weekly and monthly time-scales from SDSS, CRTS and PTF surveys are mutually compatible, as well as consistent with DRW model.




Classical model fitting involves specifying a loss function that can be used to find best-fitting parameters for a given model. A suitable metric, such as $\chi^{2}$, describes the scatter of data around the model. Choosing a better of two models involves comparing the values of the loss function. A different approach is to simultaneously find the maximum a-posteriori solution for a model that combines the two. Such possibility is afforded by Gaussian Process with combination kernel,  implemented in Celerite  \citep{foreman2017}. This technique has been successfully employed by  \cite{angus2018}  to model  quasi-periodic oscillation of star spots on top of the sinusoidally varying signal due to stellar rotation. 
This could be used to distinguish the signal of periodic stars from  stochastic quasars, or to classify quasars with binary SMBH.






\section{Timeline}


%%%%%%%%%%%%%%%%%%%% REFERENCES %%%%%%%%%%%%%%%%%%
\bibliographystyle{mnras}
\bibliography{referencesMaster}

\end{document}