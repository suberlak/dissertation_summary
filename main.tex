\documentclass[modern]{aastex62}

\begin{document}
%\maketitle
\title{Dissertation Summary}
\shorttitle{Dissertation Summary} 
\author{Krzysztof Suberlak}
\affil{University of Washington, Seattle}

%\date{ \today }

\section{Scientific Context}

\subsection{Quasars as distant, powerful Active Galactic Nuclei}
Quasars are some of the most luminous sources of radiation in the Universe.  Termed initially as Quasi-Stellar Objects, they appear as unresolved stellar-like point sources when observed with ground-based angular resolution (of the order an arcsec). However, spectroscopic observations show non-thermal radiation unlike any star. In addition, high angular resolution observations, such as the Hubble Space Telescope imaging, reveal a faint glow of a host galaxy around the intense still-unresolved source of light. The host galaxy's faint glow is so much dimmer and more diffuse than the quasar light that it took decades after the serendipitous discovery of quasars by Maarten Schmidt in 1963 \citep{schmidt1963, richards2009a} to reliably prove that quasars ``live'' in galaxies \citep{hooper1997, boyce1999, lehnert1999}. Early on it was realized that the main source of quasar light is a hot accretion disk of material infalling onto a supermassive black hole (SMBH) in the center of a host galaxy \citep{oke1965, burbidge1967}. Active Galactic Nuclei (AGN) are galaxies with signs of nuclear activity due to accretion of gas onto the central SMBH \citep{netzer2013}. When the Universe was denser at earlier times (at higher redshifts), the mergers of galaxies were more frequent, and the supply of gas higher -  thus we would expect that the AGNs of the past (or earlier times, higher redshifts), were more powerful than those at work ``today'', or in the nearby universe. This is indeed the case - we find that quasars are just much more distant, more luminous AGNs. It is only the sheer distance (and intrinsic brightness) that resulted in different features observed at first (eg. point-like accretion disk emission for a quasar, rather than the extended emission from AGN's radio lobes).  

Quasars are very interesting objects in their own right. An accretion disk is an extreme environment: it is a disk of viscously heated matter inspiralling onto an SMBH \citep{ruan2017}. The disk by itself is merely a few times bigger than the solar system - \cite{mudd2017} finds with reverberation  mapping that disk sizes for their sample of quasars are between 1-10 light days, that is ten to a hundred times the size of the orbit of Pluto, at 0.23 light days. 
The accretion disk due to a relatively small size compared to the host galaxy cannot be resolved with even the largest currently operating or planned telescopes (0.01 miliarcsec at 20 Mpc is beyond the reach of James Webb Space Telescope or European Extremely Large Telescope, \cite{gardner2006,dioaliti2015}). Therefore study of the the accretion disk is achieved by microlensing \citep{agol1999, jimenez2012} and reverberation mapping \citep{jiang2017, mudd2017}.  Emission from the accretion disk overwhelms all the light from the stars of the host galaxy, but with sufficient light collecting area and exposure it is possible to image the host too \citep{hutchings2002,kotilainen2013, falomo2014, liuzzo2016, bayliss2017}. 

\subsection{Scientific uses of quasars}
Quasars are an excellent probe of the evolution of the universe over time. Present across vast range of distances (redshifts), quasars are witnesses to the conditions in the universe from the earliest times of first galaxies, to later galaxy mergers.  The distribution of quasars peaks at the redshift $\approx$2, and they have been seen all the way from redshift 0.1 to 7.5 \citep{paris2017, banados2018}. Before it reaches us, quasar light passes through other galaxies and clouds of gas, which leave their imprint  in the form of absorption lines and troughs.  For this reason quasar spectra are useful  for studies of the intergalactic medium (\cite{prochaska2014} and references therein),  He II reionization history \citep{khrykin2017}, Damped Lyman $\alpha$ Absorbers \citep{wolfe2005,murphy2016,parks2018}.  Quasars quite literally shine light on the epoch of reionization - the moment when the universe, full of neutral Hydrogen and Helium from the Big Bang, became ionized \citep{glikman2011, masters2012, ross2013}. Counting the number of quasars as a function of luminosity yields the Quasar Luminosity Function (QLF), which is a diagnostic of the SMBH growth rate, quasar activity, and the evolution of the galactic Initial Mass Function \citep{schawinski2012,mcgreer2013,alsayyad2016}. 



\subsection{Quasars - objects variable on multiple timescales }
Given the inherent dynamics of the quasar phase of galaxy's lifetime, it makes sense that as the gas supply to the SMBH environment varies, the brightness should change over time both for an accretion disk, as well as any absorption/emission features caused by the orbiting clouds \citep{stern2017, schawinski2015}.  Since for a quasar the emission from the accretion disk is dominant compared to the stellar light from the host galaxy, we can assume that all the coherent variability  in the quasar light curve is related to the accretion disk variability. Quasar brightness varies over timescales from millions of years to hours, with variability on each timescale tied to a different physical mechanism \citep{sartori2018}.  The longest timescales of millions of years (postulated but not directly measured) correspond to the changes in overall accretion pattern, for instance as the galactic merger that started the quasar phase progresses.  Indeed, it is now thought that quasars are just an episode (or multiple episodes) in life of almost any galaxy \citep{alexander2012, kormendy2013}. Progressing to shorter (more directly measurable) timescales of years, quasar brightness may change due to a variable strength of an underlying continuum emission, tied to the amount of available gas (see eg. Changing-Look AGNs, and discovered recently Changing-Look Quasars - \cite{ruan2016, macleod2016, graham2017}). There is also variability on medium timescales of weeks to months that can be easily observed in a quasar light curve. Historically it was variability on these medium timescales  in the optical light that first hinted at the relatively small size of the region emitting the light (if the brightness of an entire disk changes, two parts of the disk must be separated by less than the light travel time -  \cite{mudd2017, blackburne2011, morgan2010}). 

\subsection{Mathematical description of quasar light curves}
Quasar light curve in the optical wavelength looks like a random noise - there seems to be no overall pattern underlying the changes in brightness. After power spectrum analysis it turns out that the signal is stochastic - close to chaotic, but with an inherent order. Stochastic variability can be characteristic of thermal processes, such as spread of instabilities in the disk - ``hot spots'', driven by the magnetohydrodynamic (MHD) processes \citep{kelly2007, dexter2011, zu2013, kozlowski2016a}. Therefore, parameters describing quasar variability on these medium-long timescales relate to the physical accretion parameters such as disk extent, its viscosity and temperature, strength of magnetic field. On shortest timescales of hours, only recently probed by Kepler, the light curves are also consistent with  the MHD instabilities \citep{kasliwal2015a, aranzana2018, smith2018}. Recent reverberation mapping studies largely confirm this picture \citep{sun2015}. One way of describing the variation in brightness over time is to plot the signal against time - that is the classic light curve.  A light curve can be further characterized by a structure function (SF), that describes the root-mean-square scatter in magnitude differences between pairs of points as a function of time separation \citep{graham2014}. SF tells us how much correlation there is between different parts of light curve, and it is also related to the power spectral density. SF can be seen as a mirror image of the Power Spectral Density (PSD), in a sense that while SF is a function of time differences, PSD is a function of frequency - time inverse. Processes that have a flat PSD - having an equal power across the entire frequency spectrum -  are described as white noise, like white light that includes all colors. To further this analogy, stochastic processes that can be described by a simple power law, and yet depart from white noise, are termed  ``colored' noise, depending on the relative amplitude of PSD at various frequencies. Thus we have red, pink, or blue noise   (see Appendix  C in \cite{kasliwal2017}). Quasar light curves are consistent with Damped Random Walk model, or the PSD $\propto  f^{-2}$, flattening at frequencies lower than that corresponding to the characteristic timescale $\tau$ \citep{macleod2010,zu2013}.

\subsection{Quasar selection techniques}
Over the last decade new variability-based classification techniques have yielded detection of an unprecedented number of quasars allowing a population statistical study \citep{fan2001, richards2006, kozlowski2010, palanque2011, macleod2011, graham2014, alsayyad2016, ruan2017}. Classically quasars were selected using color cuts,  which mainly targets quasars at redshifts around 0.5 \textless  z \textless 2.5 (Large Bright Quasar Survey, \cite{hewett1995}, the 2dF QSO Redshift Survey \cite{croom2004}, and the 2dF-SDSS LRG and QSO Survey \cite{croom2009}).  In addition to color space, SDSS I,II selected quasars via radio matches to the FIRST survey, or x-ray matches to ROSAT survey \citep{myers2015}. Since then other classification methods have been tested, including radio source matching \citep{mcgreer2009}, near-IR color cuts \citep{banerji2012},  radio + near-IR \citep{glikman2012}, mid-IR \citep{stern2005, richards2009a, stern2012}, X-rays \citep{trichas2012},  midIR + X-rays \citep{lacy2004, hickox2007, hickox2009}, variability \citep{schmidt2010, butler2011, macleod2011, palanque2011,palanque2016}. There are some studies that combined color and variability information \citep{tie2017, peters2015, sesar2007}. 

\subsection{Variability with Damped Random Walk model}
A better understanding of how well the Damped Random Walk model can be used to model quasar variability is crucial for using the DRW best-fit parameters to characterize and classify quasars. It is a relatively recent field : \cite{kelly2009} showed that DRW parameters can be linked to the physics of the accretion disk. Since \cite{kozlowski2010} proposed the DRW model for quasar selection, \cite{macleod2010} moved the field of quasar variability studies from a handful of objects to a proper statistical study fitting over 9 000 quasars in SDSS Stripe 82 with the DRW model.  \cite{kozlowski2017a} further showed that when using the DRW model, the length of light curve is the most important predictor of biases for the best-fit parameters. His main conclusion is that when the light curve length (baseline) is insufficient (less than 10 times the intrinsic timescale ), then the bias will be larger than a few \%; for a light curve of length similar to the intrinsic timescale, we expect a bias of a factor of 3 in the retrieved timescale (confirmed by our independent study). 


\section{Proposed Research}

There are several crucial questions related to quasar variability that remain open. First, despite general agreement that the DRW is a good first-order description of quasar variability (\cite{zu2011,kozlowski2010}), there have been some inquiries to the fidelity of that description on all timescales \cite{zu2013, kasliwal2017, sartori2018}. In particular, \cite{graham2014} reported  a detection of characteristic rest-frame timescale of 54 days, which has not been detected before. It was surprising given that other studies more well tuned to short timescales did not report such findings ( especially studies involving Kepler K2  AGN light curves with extremely good cadence - see \cite{kasliwal2015a, aranzana2018, smith2018}). We aim to reconcile the Catalina Real-Time Transient Survey (CRTS) data with the Sloan Digital Sky Survey (SDSS) and Palomar Transient Factory (PTF) data for the same SDSS Stripe 82 quasars. 

Second, although \cite{kelly2011} showed that the DRW is a good description of the quasar variability, \cite{kozlowski2017a} suggested that inherent to DRW description are biases in retrieved parameters dependent on light curve length. We want to evaluate the existing tools for fitting light curves (eg. the implementation of \cite{rybicki1995} method in JAVELIN \cite{zu2011}, George - \cite{ambikasaran2015}, C\'el\'erit\'e - \cite{foreman2017}), and find the best tool for future studies given their scalability (speed), memory usage, bias, to be prepared for fitting of billions of light curves in the LSST era. 

Finally, we consider whether we can improve the ``classical'' method of light curve classification using combination kernel Gaussian Processes. In the standard paradigm,  to compare models A vs B (eg. sinusoidal oscillation vs DRW), one needs to fit (optimize the cost function) for model A and B,  calculate metric describing the ``goodness of fit'' (eg. $\chi^{2}_{DOF}$), and then choose a better fitting model based on that metric. We investigate a Bayesian method of fitting to the data a combination kernel Gaussian process, where one obtains simultaneously maximum a-posteriori (MAP) estimates for amplitudes of component kernels. It has the potential advantage of fitting amplitudes for the two models in one pass, enabling quicker model selection, crucial in preparation for large, synoptic sky surveys (ZTF,LSST).

\subsection{Discrepant quasar timescales}
To answer the first question, we presented in \cite{suberlak2017} an improved photometric error analysis for the 7 100 CRTS optical light curves for quasars from the SDSS Stripe 82 catalog. The SDSS imaging survey has provided a time-resolved photometric data set, which greatly improved our understanding of the quasar optical continuum variability. Data for monthly and longer time-scales are consistent with a damped random walk. Newer data obtained by CRTS provided puzzling evidence for enhanced variability, compared to SDSS results, on monthly time-scales. Quantitatively, SDSS results predicted about 0.06 mag root-mean-square (rms) variability for monthly time-scales, while CRTS data showed about a factor of 2 larger rms, for spectroscopically confirmed SDSS quasars. Our analysis has successfully resolved this discrepancy as due to slightly underestimated photometric uncertainties from the CRTS image processing pipelines. As a result, we found that the correction for observational noise was too small and the implied quasar variability was too large. The CRTS photometric error correction factors, derived from detailed analysis of non-variable SDSS standard stars that were re-observed by CRTS, are about $20 \mathrm{-} 30 \%$, and result in reconciling quasar variability behavior implied by the CRTS data with earlier SDSS results. An additional analysis based on independent light curve data for the same objects obtained by the Palomar Transient Factory provides further supported for this conclusion. In summary, the quasar variability constraints on weekly and monthly time-scales from SDSS, CRTS and PTF surveys are mutually compatible, as well as consistent with DRW model, as described in \cite{suberlak2017}. 



\subsection{Extending the light curve baseline, using the right tools}
We first confirm the \cite{kozlowski2017a} scaling relations by testing the retrieval of simulated light curve parameters ($\sigma$ related to asymptotic amplitude $SF_{\infty}$, and characteristic timescale $\tau$) with C\'el\'erit\'e.  We find that a longer light curve baseline can decrease the bias in retrieved input timescale $\tau$, which implies that adding new data to quasar light curves can help improve the results of \cite{macleod2011}.  Using a fixed light curve length of $t_{exp} = 8$ years we simulate 100 different input time scales $\tau_{in}$ ,  where  $\rho = \tau_{in} / t_{exp} \in   { 0.01 : 15}$.  The bias is not affected by different cadence (we tested SDSS and CRTS cadences of N=80 or N=445 points).
Encouraged by this result we add more points to quasar light curves, and revisit relations studied by  \cite{macleod2011} and \cite{hernitschek2016}. We use PanSTARRS \citep{chambers2011} DR2 \citep{flewelling2018}, CRTS DR2 \citep{drake2009}, and PTF \citep{rau2009}. Starting from the SDSS  DR7 near-simultaneous ugriz photometry for Stripe 82 quasars \citep{schneider2008}, we cross-match the catalogs. We find a common photometric solution using the S82 standard stars (PTF and CRTS use white light - see \cite{djorgovski2011a}). We validate data quality by comparing mean CRTS magnitude vs mean SDSS magnitude, and mean PS1 g-band to mean PS1 r-band. 

To verify the improvement in retrieved DRW parameters we first simulate 10 000 DRW light curves at the SDSS cadence, and we test how adding more data (corresponding to PS1, PTF, CRTS data points) affects the fidelity of the retrieved parameters to the input values. Confirming that adding more data at these cadences improves the fit, we fit the real data with the DRW model, and using the spectroscopic information about the quasars (such as black hole mass, luminosity - \cite{kelly2013}) we reconsider relations tested by \cite{macleod2011}.  We also test how selecting photometry from only a subset of surveys ($\tau_{PS1}$, $\tau_{(PS1+SDSS)}$,  $\tau_{SDSS}$), or selecting only bright quasars  ($\tau_{\langle mag\rangle<19}$), affects the fidelity of retrieved DRW parameters.

\subsection{Improving the classical model fitting with Gaussian processes}

Classical model selection involves specifying a loss function that can be used to find best-fitting parameters for a given model. A suitable metric, such as $\chi^{2}$, is used to  describe the scatter of data around the model. Choosing a better model involves comparing the values of the loss function. A different approach is to simultaneously find the maximum a-posteriori solution for a model that combines the two. Such possibility is afforded by Gaussian Process with combination kernels,  implemented in C\'el\'erit\'e  \citep{foreman2017}. For the last decade Gaussian Processes have become more well known, employing a class of functions that  are characterized by covariance between pairs of points in the dataset \citep{foreman2017}.  The combination kernel technique has been employed by  \cite{angus2018}  to model  quasi-periodic oscillation of star spots on top of the sinusoidally varying signal due to stellar rotation. Success of this approach furthers our interest in employing this methodology to mine the data for a sinusoidal signal superimposed on top of the DRW, such as is the case for the binary SMBH \citep{charisi2018}. 

We first test the combination kernel method, simulating the DRW light curve with added sinusoidal modulation (parametrized by amplitude A, period P). With input characteristic timescale of $\tau  = 100 $ days, and regular sampling every dt = 5 days  , we explore regimes from $A \ll SF_{\infty}$,  to $A \approx SF_{\infty}$ , and from $P \ll \tau$, to $P \gg \tau$. We aim to find whether we can distinguish between a pure stochastic signal, and stochastic signal with sinusoidal modulation. We then test the combination kernel Gaussian process on spectroscopically confirmed SDSS S82 quasars, and finally find  the combination kernel amplitudes for all SDSS S82 light curves.


\section{Timeline}
\begin{itemize}
    \item Project ``Discrepant quasar timescales''
     \begin{itemize}
     	\item published in \cite{suberlak2017}
     \end{itemize}

    \item Project ``Improved constraints on quasar variability time scales''
    \begin{itemize}
        \item Completed: identify an adequate fitting tool and confirm results about biased
  best-fit parameters from Kozlowski (2017)
        \item Complete by May 15, 2018: collate the SDSS, Pan-STARRS, CRTS and PTF
  light curve data for $\approx$9,000 quasars from the SDSS Stripe 82 analyzed in 
  the MacLeod et al. paper series
        \item Complete by May 30, 2018: complete C\'el\'erit\'e-based DRW fits
        \item Complete by June 30, 2018: analyze results and produce a rough paper outline
        \item Complete by Aug 31, 2018: produce the first draft of the paper  
        \item Complete by Sep 30, 2018: submit the second paper  
     \end{itemize}
    \item Project ``Quasar/star light-curve based classification using Gaussian processes''
    \begin{itemize}
    	\item Complete by Dec 1, 2019: complete testing of the combination kernel method on simulated light curves 
    	\item Complete by Jan 1, 2019: complete verification of the method on spectroscopically confirmed SDSS Stripe 82 quasars
    	\item Complete by Feb 1, 2019: complete combination kernel amplitudes for SDSS Stripe 82 objects
    	\item Complete by March 1, 2019: produce the first draft
    	\item Complete by Apr 1, 2019: submit the third paper 
    \end{itemize}

 \end{itemize}


%%%%%%%%%%%%%%%%%%%% REFERENCES %%%%%%%%%%%%%%%%%%
\bibliographystyle{mnras}
\bibliography{referencesMaster}

\end{document}